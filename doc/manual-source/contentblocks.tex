%Copyright 2011 Newcastle University
%
%   Licensed under the Apache License, Version 2.0 (the "License");
%   you may not use this file except in compliance with the License.
%   You may obtain a copy of the License at
%
%       http://www.apache.org/licenses/LICENSE-2.0
%
%   Unless required by applicable law or agreed to in writing, software
%   distributed under the License is distributed on an "AS IS" BASIS,
%   WITHOUT WARRANTIES OR CONDITIONS OF ANY KIND, either express or implied.
%   See the License for the specific language governing permissions and
%   limitations under the License.

\begin{chapter}{\label{cha:content_blocks}Content blocks}
  A number of the properties within the exam objects are \codecontent{content}
  blocks, marked as \emph{(Content)} in the various tables throughout this
  manual. These blocks are displayed as HTML when the exam is run, so any valid
  HTML is allowed.  
  
  It is not necessary for you to be familiar with HTML markup --- simpler
  \emph{Textile} markup (\url{http://textile.thresholdstate.com/}) is also
  permitted, and its use is encouraged.  Only use HTML as a last resort, \eg
  when you want to include a video in your exam.  See the Textile website for
  an explanation of the markup.

  When you want to write more complicated mathematics, the restrictions of HTML
  will be too great.  This problem is overcome by using \LaTeX\ for displaying
  mathematics, as explained in the next section.

  \section{\label{sec:latex_formatting}\LaTeX}
  It is possible to use \LaTeX\ syntax to display mathematics in
  \codecontent{content} blocks, \eg in question \codeobject{statement}s, or
  part \codeobject{prompt}s.  
  
  Inline math-mode \LaTeX\ can be used by enclosing content in dollar signs;
  display-mode \LaTeX\ (\ie on its own line, and in a larger font) can be
  achieved by enclosing content between escaped square brackets (\verb"\[" and
  \verb"\]").

  \subsection{Variables}
  In \S\ref{sec:using_variables} we explained how to use declared variables in
  \codecontent{content} blocks and part answers.  If you would like to use
  \LaTeX\ to format content, then it is not possible to use the double-brace
  syntax within this content, because the braces conflict with \LaTeX's syntax.

  Instead, an entirely equivalent syntax is to use the \verb"\var{}" command.
  Just as with braces, using \verb"\var{}" will evaluate its argument according
  to any declared variables and substitute in the value.  See
  table~\ref{tab:latex_var} for some examples.
  %
  \begin{table}[ht]
    \centering
    \begin{tabular}{ll}
      \hline
      Brace syntax         & \LaTeX\ syntax \\
      \hline
      \verb"What is {a}+{b}?"     & \verb"What is $\var{a}+\var{b}$?" \\
      \verb"Calculate 2{a}-{b^2}" & \verb"Calculate $2\var{a}-\var{b^2}$" \\
      \hline\hline
    \end{tabular}
    \caption{\label{tab:latex_var}
      Examples of variable usage within \LaTeX\ content.
    }
  \end{table}

  Using \LaTeX\ will allow you to display much more complicated mathematics
  than is possible with raw HTML.  Questions involving differentiation or
  integration, for example, become easy to write, as shown in
  table~\ref{tab:more_complicated_latex}.
  %
  \begin{table}[ht]
    \centering
    \begin{tabular}{ll}
      \hline
      Content & Display \\
      \hline
      \rule{0em}{5ex}\verb"\[\frac{\partial}{\partial x} x^{\var{a}}y^{\var{b}}\]" & $\displaystyle{\frac{\partial}{\partial x} x^{2}y^{5}}$ \\[2ex]
      \rule{0em}{5ex}\verb"\[\int_{\var{a}}^{\var{b}}{x^{4}\mathrm{d}x}\]" &
      $\displaystyle{\int_{2}^{5}{x^{4}\mathrm{d}x}}$ \\[2ex]
      \hline\hline
    \end{tabular}
    \caption{\label{tab:more_complicated_latex}
      Examples of variable usage within more complicated \LaTeX\ content.
      Variables are declared as \texttt{a: 2}, \texttt{b: 5}.
    }
  \end{table}

  \subsection{\label{sec:simplification}Simplification}
  One of the most powerful features of \numbas\ is the ability to
  automatically simplify and rearrange expressions according to a set of
  ``simplification rules.''  Examples of when this might be useful include
  cancelling numerical factors in fractions, collecting numerical factors
  together, simplifying expressions involving 0 or 1, etc.  Since you do not
  necessarily know what values your random variables will take, the ability to
  automatically simplify expressions is very useful.

  Simplification is performed in \LaTeX\ content with the \verb"\simplify{}"
  command.  Simplification can also be performed on the displayed answers in a
  JME part, using the \codeprop{answersimplification} property of the JME part
  (see \S\ref{sec:jme_part} for more information on the JME part type).  The
  syntax of the simplification command is
  %
  \begin{Verbatim}
    \simplify[rules]{expression}
  \end{Verbatim}
  %
  where \verb"expression" is the mathematical expression you want to simplify,
  and the optional \verb"rules" argument further refines how the expression
  should be simplified.

  As an example, consider the code \verb"$\simplify{ ({a}*x)/({b}*y) }$".
  Assuming $a=2$ and $b=-1$, the result will be displayed as
  \verb"$\frac{-2x}{y}$".

  The \verb"rules" argument is a string of 16 ones and zeroes, where each
  digit specifies whether a particular simplification rule should be turned on
  or off.  The default is for all rules to be turned on, so the example above
  is equivalent to
  %
  \begin{Verbatim}
    $\simplify[1111111111111111]{ ({a}*x)/({b}*y) }$
  \end{Verbatim}
  %
  The full list of simplification rules is given in
  table~\ref{tab:simplification_rules}.
  %
  \begin{table}[ht]
    \centering
    \begin{tabular}{ll}
      \hline
      Rule name        & Meaning \\
      \hline
      unitFactor       & cancel \verb"1*x" to \verb"x" \\
      unitPower        & cancel \verb"x^1" to \verb"x" \\
      unitDenominator  & cancel \verb"x/1" to \verb"x" \\
      zeroFactor       & cancel \verb"0*x" to \verb"0" \\
      zeroTerm         & cancel \verb"x+0" to \verb"x" \\
      zeroPower        & cancel \verb"x^0" to \verb"1" \\
      collectNumbers   & collect \verb"1*2*3" to \verb"6" \\
      simplifyFractions& cancel \verb"(a*b)/(a*c)" to \verb"b/c" \\
      zeroBase         & cancel \verb"0^x" to \verb"0" \\
      constantsFirst   & rearrange \verb"x*3" to \verb"3*x" \\
      sqrtProduct      & simplify \verb"sqrt(a)*sqrt(b)" to \verb"sqrt(a*b)" \\
      sqrtDivision     & simplify \verb"sqrt(a)/sqrt(b)" to \verb"sqrt(a/b)" \\
      sqrtSquare       & simplify \verb"sqrt(x^2)" to \verb"x" \\
      trig             & simplify various trigonometric values \eg
      \verb"sin(n*pi)" to \verb"0" \\
      otherNumbers     & simplify \verb"2^3" to \verb"8" \\
      fractionNumbers  & display all numbers as fractions instead of decimals \\
      \hline\hline
    \end{tabular}
    \caption{\label{tab:simplification_rules}
      The available simplification rules, in order.  Note that the rule names
      are case-sensitive, and a rule will not be applied if it does not appear
      exactly as in the table.
    }
  \end{table}

  An alternative to specifying the entire string of ones and zeroes is to
  explicitly name the rules you would like to be turned on, \eg
  %
  \begin{Verbatim}
    $\simplify[constantsFirst,zeroPower]{expression}$
  \end{Verbatim}
  %
  Once an expression has been simplified in this way, it is converted to
  \LaTeX\ automatically.

\end{chapter}
