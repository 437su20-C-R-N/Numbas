%Copyright 2011 Newcastle University
%
%   Licensed under the Apache License, Version 2.0 (the "License");
%   you may not use this file except in compliance with the License.
%   You may obtain a copy of the License at
%
%       http://www.apache.org/licenses/LICENSE-2.0
%
%   Unless required by applicable law or agreed to in writing, software
%   distributed under the License is distributed on an "AS IS" BASIS,
%   WITHOUT WARRANTIES OR CONDITIONS OF ANY KIND, either express or implied.
%   See the License for the specific language governing permissions and
%   limitations under the License.

\begin{chapter}{\label{cha:question_parts}Question parts and types}
  Each exam consists of a number of questions which, in turn, consist of a
  number of parts.  A part will generally be where the student will need to
  provide some input as their answer.

  \numbas\ supports a variety of part types.  This chapter describes each
  part type in detail, explaining under which circumstances you might use each
  type, and how to define the properties of a \codeobject{part} object.
  
  Each \codeobject{part} object is an element of the \verb"parts" array, within
  a \codeobject{question} object.  Parts will typically be defined as follows
  (see, for example, listing~\ref{lst:example}):
  %
  \begin{Verbatim}
    parts: [
      {
        type: <parttype>
        <other_properties>
      }
    ]
  \end{Verbatim}
  %
  The \codeprop{type} property is mandatory; other properties are optional.
  Table~\ref{tab:part_object} lists the \codeobject{part} object properties
  which can be used independently of the part type.  Properties which are
  specific to particular part types are explained in the following sections.

  \section{\label{sec:number_entry_part}Number entry}
  In a \emph{Number entry} part, the student's answer should be a number
  (integer or real), which either matches the correct answer exactly, or lies
  within a particular range of the correct answer.  The correct answer cannot
  be a fraction or a complex number; use the JME part type for these cases ---
  see \S\ref{sec:jme_part}.

  For a number entry part, set \codeprop{type} to \verb"numberentry".
  Table~\ref{tab:number_entry_part} lists the properties of this part type.
  (All properties are optional, unless otherwise stated, and if not set, will
  take the default value listed.)
  %
  \begin{table}[ht]
    \centering
    \begin{tabular}{lp{20em}l}
      \hline
      Property & Description & Default value \\
      \hline
      \codeprop{minvalue}* & The student's answer must be greater than or equal
      to this value for the answer to be considered correct. & \verb"0" \\
      \codeprop{maxvalue}* & The student's answer must be less than or equal to
      this value for the answer to be considered correct. & \verb"0" \\
      \codeprop{answer}* & The student's answer must be exactly equal to this
      value for the answer to be considered correct. & \verb"0" \\
      \codeprop{integeranswer} & This specifies whether the answer must be an
      integer. & \verb"false" \\
      \codeprop{partialcredit} & This specifies the percentage of the total
      part mark to be awarded, if \codeprop{integeranswer} is set, and the
      student's answer is within the allowable range, but not an integer. &
      \verb"0" \\
			\codeprop{inputStep} & The amount to change the entered number by when the student clicks on the up or down arrows in the input field. & \verb"1" \\
      \hline\hline
    \end{tabular}
    \caption{\label{tab:number_entry_part}
      The valid properties of a number entry part.  *Note that if
      \codeprop{answer} is set, then \codeprop{minvalue} and
      \codeprop{maxvalue} need not be.  If \codeprop{answer} is not set, then
      both \codeprop{minvalue} and \codeprop{maxvalue} must be.
    }
  \end{table}

  Declared variables and functions can be used in the \codeprop{answer},
  \codeprop{minvalue}, and \codeprop{maxvalue} properties, and the result of
  any variable evaluation must be a pure number (integer or real).

  The example exam (listing~\ref{lst:example}) includes two number entry parts,
  where exact answers are required.  Another example of a number entry part is
  the following:
  %
  \begin{Verbatim}
    {
      type: numberentry
      marks: 1
      prompt: "What is $\sqrt{2}$?"
      minvalue: 1.41
      maxvalue: 1.415
    }
  \end{Verbatim}
  %
  Here, the student's answer must lie in the range $[1.41,1.415]$ for it to be
  marked correct, and one mark awarded.

  \section{\label{sec:jme_part}Judged mathematical expression}
  In a \emph{Judged Mathematical Expression (JME)} part, the student must enter
  a mathematical expression, which is equivalent to an expression defined to be
  correct.  You should use this part type when the answer needs to be a
  mathematical expression, \eg rearranging algebra, integrating a function,
  stating an identity, or when you need to use complex numbers.

  For a JME part, set the part \codeprop{type} to \verb"jme".
  Table~\ref{tab:jme_object} lists the properties of the JME part object, and
  table~\ref{tab:restriction_object} lists the properties of the
  \codeobject{restriction} object, which is used to define constraints on the
  student's answer in a JME part.  How this object is used is described later.

  Subsequent sections will describe various aspects of the JME part in greater
  detail.
  %
  \begin{table}[ht]
    \centering
    \begin{tabular}{lp{18em}l}
      \hline
      Property & Description & Default value \\
      \hline
      \codepropreq{answer} & A JME expression defining the correct answer (see
      chapter~\ref{cha:jme_syntax}). & \nodef \\
      \footnotesize{\codeprop{answersimplification}} & The simplification rules which should
      be applied to the revealed answer.  For display purposes only.  A string
      of 0s and 1s, or the names of simplification rules to enable & \mbox{see
      \S\ref{sec:answer_simplification}}. \\
      \codeprop{checkingtype} & The method to use when comparing the student's
      answer to the correct answer (see \S\ref{sec:answer_comparison}). &
      \verb"reldiff" \\
      \codeprop{checkingaccuracy} & The inaccuracy tolerance when checking the
      student's answer against the correct answer.  What value this takes
      depends on the \codeprop{checkingtype} (see
      \S\ref{sec:answer_comparison}). & \verb"0" \\
      \codeprop{failurerate} & The number of comparison failures to allow,
      before the student's answer is deemed incorrect (see
      \S\ref{sec:answer_comparison}). & \verb"1" \\
      \codeprop{vsetrangepoints} & The number of values to randomly choose from
      \codeprop{vsetrange}, for each unknown, when evaluating variables in an
      answer comparison (see \S\ref{sec:answer_comparison}). & \verb"5" \\
      \codeprop{vsetrange} & A 2-element array defining the start and end
      points from which \codeprop{vsetrangepoints} values should be chosen, in
      an answer comparison (see \S\ref{sec:answer_comparison}). &
      \verb"[0,1]" \\
      \codeprop{maxlength} & A \codeobject{restriction} object defining the
      maximum length, in characters, which a student's answer can be, or zero
      if there should be no restrictions (see
      \S\ref{sec:length_restrictions}).  & \verb"{}" \\
      \codeprop{minlength} & A \codeobject{restriction} object defining the
      minimum length, in characters, which a student's answer can be, or zero
      if there should be no restrictions (see
      \S\ref{sec:length_restrictions}). & \verb"{}" \\
      \codeprop{musthave} & A \codeobject{restriction} object defining the
      \codeprop{strings} which a student's answer must contain (see
      \S\ref{sec:string_restrictions}). & \verb"{}" \\
      \codeprop{notallowed} & A \codeobject{restriction} object defining the
      \codeprop{strings} which a student's answer must not contain (see
      \S\ref{sec:string_restrictions}). & \verb"{}" \\
      \hline\hline
    \end{tabular}
    \caption{\label{tab:jme_object}
    The valid properties of a JME object.  The \codeprop{answer} property is
    required, as denoted by the italic typeface.
    }
  \end{table}
  %
  \begin{table}[ht]
    \centering
    \begin{tabular}{lp{20em}l}
      \hline
      Property & Description & Default value \\
      \hline
      \codeprop{message} & \emph{(Content)} A message to display to the
      student, when this particular restriction is violated. & \nodef \\
      \codeprop{partialcredit} & Percentage of the available marks to award, if
      this restriction is violated.  Note that \codeprop{partialcredit} is
      compounded if multiple restrictions are in use. & \verb"0" \\
      \codeprop{length} & If this is a length restriction, then the length
      restriction of the student's answer, in characters. & \verb"0" \\
      \codeprop{strings} & If this is a string restriction, then an array of
      strings, which must, or must not, be included in the student's answer. &
      \verb"[]" \\
      \codeprop{showstrings} & If this is a string restriction, determines
      whether the list of \codeprop{strings} is displayed to the student, if
      \codeprop{strings} is set, and the student violates the string
      restriction. & \verb"false" \\
      \hline\hline
    \end{tabular}
    \caption{\label{tab:restriction_object}
    The valid properties of a restriction object.  The \codeprop{message}
    property is a \codecontent{content} block --- see
    chapter~\ref{cha:content_blocks}.
    }
  \end{table}

  \subsection{\label{sec:answer_comparison}Answer comparison}
  To determine whether the student's answer matches the correct answer, both
  expressions are evaluated several times, by substituting a set of randomly
  chosen values for the unknowns in each expression.  These randomly chosen
  values are controlled by the \codeprop{vsetrangepoints} and
  \codeprop{vsetrange} properties.

  As an example, suppose that the correct answer were $3x^{2}$, where the numbers 3 and 2
  may or may not have been randomly chosen.  In this expression $x$ is the only
  unknown.  The default comparison check is to randomly choose five values from
  the range $[0,1]$, and substitute them for $x$ in the expression.  The same
  is done for the student's answer (using the same five values); for each
  evaluation, the student's answer must be sufficiently close to the correct
  answer for it to be considered correct.

  Because evaluation in this way can lead to rounding errors, there are a
  number of available ``checking methods,'' to overcome any possible problems.
  You can also decide that the comparison is allowed to fail up to a certain number
  of times before the student's answer is deemed incorrect --- see
  \S\ref{sec:accuracy} for more information on this.

  A consequence of this method of comparison is that any mathematical
  expression which is equivalent to the correct answer is permitted.  For
  example, if the correct answer were $3x^{2}$ (\verb"3*x^2" in JME syntax),
  then the following student answers would all be marked correct:
  \verb"2*1.5*x^2", \verb"3*x*x", \verb"2*x^2+x^2".

  This behaviour can be prevented by restricting the student's answer in
  various ways, \eg by defining string restrictions, length restrictions, and
  accuracy settings.

  \subsubsection{\label{sec:string_restrictions}String restrictions}
  It is possible to restrict the characters and strings a student can enter as
  part of their answer.  In contrast, you can also specify that the student's
  answer must contain particular characters or strings.  This is done through
  the \codeprop{notallowed} and \codeprop{musthave} properties of the JME
  \codeobject{part} object.  These properties are \codeobject{restriction}
  objects as shown in table~\ref{tab:restriction_object}.  Consider the
  following example:
  %
  \begin{Verbatim}
    musthave: {
      strings: [hello,goodbye]
      message: Your answer is missing some words.
    }
    notallowed: {
      strings: [+,*,-]
      message: You have included forbidden characters in your answer.
      showstrings: true
    }
  \end{Verbatim}
  %
  For this particular part, the student's answer must contain the strings
  \verb"hello" and \verb"goodbye", and must not include \verb"+", \verb"*", or
  \verb"-".  If the student's answer violates any of these restrictions, then
  the appropriate message is shown, and in the \codeprop{notallowed} case, the
  student will also be shown which strings are not allowed.  In both cases the
  student would receive no marks for the part, since the
  \codeprop{partialcredit} property is set to \verb"0" by default.

  \subsubsection{\label{sec:length_restrictions}Length restrictions}
  In certain circumstances, you might decide that the student's answer must be
  shorter or longer than a set number of characters.
  
  Consider asking the student to expand $(x+1)^{2}$.  The correct answer is
  $x^{2}+2x+1$, but due to the comparison method, the student could enter the
  original expression, and be awarded the marks, since the two expressions are
  equivalent.
 
  One way to solve the problem (not necessarily the best) is to require that
  the student's answer is longer than seven characters, which is the number of
  characters required to enter \verb"(x+1)^2" (in JME syntax).  You could also
  combine the length restriction with a string restriction, forbidding left or
  right parentheses.  The JME part object would then contain the following:
  %
  \begin{Verbatim}
    minlength: {length: 8}
    notallowed: {strings: [(,)]}
  \end{Verbatim}
  %
  In this case, the other properties of the \codeobject{restriction} object
  take their default values.

  \subsubsection{\label{sec:accuracy}Accuracy}
  As briefly mentioned earlier, evaluation and comparison of the correct answer
  and student's answer can lead to rounding errors.  Several checking methods
  are available to help alleviate this problem.  The \codeprop{checkingtype}
  property controls the method, and \codeprop{checkingaccuracy} controls the
  tolerance involved in the method.  See table~\ref{tab:checking_types} for a
  list of the checking methods.
  %
  \begin{table}[ht]
    \centering
    \begin{tabular}{llp{18em}}
      \hline
      \multicolumn{3}{c}{Correct answer: $a_{c}$; Student answer: $a_{s}$} \\
      Checking method & Description & When comparison fails \\
      \hline
      \codeprop{absdiff} & \small{Absolute difference} &
      $\abs{a_{c}-a_{s}}>\codeprop{checkingaccuracy}$ \\
      \codeprop{reldiff} & \small{Relative difference} &
      $\abs{(a_{c}-a_{s})/a_{s}}>\codeprop{checkingaccuracy}$ \\
      \codeprop{dp} & \small{Decimal points} & $a_{c}$ and $a_{s}$ rounded to
      \codeprop{checkingaccuracy} decimal points.  Failure if the rounded
      numbers are not exactly equal. \\
      \codeprop{sigfig} & \small{Significant figures} & $a_{c}$ and $a_{s}$
      rounded to \codeprop{checkingaccuracy} significant figures.  Failure if
      the rounded numbers are not exactly equal. \\
      \hline\hline
    \end{tabular}
    \caption{\label{tab:checking_types}
      The available checking methods, and the methods for determining when the
      comparison between the correct answer and student's answer fails.
    }
  \end{table}

  The \codeprop{failurerate} property can be set, so that the comparison is
  allowed to fail a certain number of times before the student's answer is
  deemed incorrect.  You might do this if you know the comparison will probably
  fail so many times over the range \codeprop{vsetrange}.

  Finally, be wary of setting a comparison range including points on which
  the correct answer is not defined.  For example, if the correct answer were
  $1/(2x-1)$, then the default comparison range of $[0,1]$ is not acceptable,
  because the expression is not defined on $x=1/2$.

  \subsection{\label{sec:answer_simplification}Revealed answer simplification}
  The simplification rules applied to the displayed answer in a JME part can be
  specified using the \codeprop{answersimplification} property of the JME part.
  The available rules are identical to those explained in
  \S\ref{sec:simplification}.  The property itself is defined as
  %
  \begin{Verbatim}
    answersimplification: 1111111011111011
    answersimplification: "sqrtProduct,unitPower"
  \end{Verbatim}
  %
  where the string of 1s and 0s shown here defines the default set of rules.

  \subsection{JME example}
  The following example of a JME question part brings together a number of the
  topics explained in the preceding sections.
  %
  \begin{Verbatim}
  {
    type: jme
    marks: 2
    prompt: "What is the length of the vector $(x,y)$?"
    answer: sqrt(x^2+y^2)
    checkingtype: absdiff
    checkingaccuracy: 0.01
    failurerate: 1
    vsetrangepoints: 5
    vsetrange: [-10,10]
    maxlength: {
      length: 20
      partialcredit: 50
      message: Your answer shouldn't be very complicated.
    }
    notallowed: {
      strings: [-,/]
      message: Please write your answer in the usual way.
    }
  }
  \end{Verbatim}
  %
  This JME part asks the question ``What is the length of the vector
  $(x,y)$?'', for which the correct answer is $\sqrt{x^{2}+y^{2}}$.  Two marks
  are awarded for a correct answer.  The absolute difference method of
  comparison will be used; this difference should be less than $0.01$.  If any
  comparison does not match, the student's answer is deemed incorrect.  Five
  points are randomly chosen over the range $[-10,10]$ for the comparison.

  The student's answer can be no longer than 20 characters; if it is, then a
  message is shown, and only 50\% of marks will be awarded for an otherwise
  correct answer.

  The student's answer cannot include \verb"-" or \verb"/"; if it does, then a
  message is shown, and no marks are awarded.

  \section{\label{sec:pattern_match_part}Pattern match}
  In a \emph{Pattern match} part the student must enter a string which matches
  a regular expression.  The valid properties of the pattern match part are
  given in table~\ref{tab:pattern_match_part}.
  %
  \begin{table}[ht]
    \centering
    \begin{tabular}{lp{20em}l}
      \hline
      Property & Description & Default value \\
      \hline
      \codepropreq{answer} & A regular expression against which to test the
      student answer. & \nodef \\
      \codepropreq{displayanswer} & \emph{(Content)} The string to display as
      the correct answer, when the correct answer is revealed. & \nodef \\
      \codeprop{casesensitive} & Whether the use of upper or lower case
      characters matters. & \verb"false" \\
      \codeprop{partialcredit} & The percentage of the original part mark to
      award, if \codeprop{casesensitive} is \verb"true", and the student is
      only incorrect because of a difference in character case. & \verb"0" \\
      \hline\hline
    \end{tabular}
    \caption{\label{tab:pattern_match_part}
      The valid properties of a pattern match part.  The \codeprop{answer} and
      \codeprop{displayanswer} properties are required, as denoted by the
      italic typeface.  In addition \codeprop{displayanswer} is a
      \codecontent{content} block --- see chapter~\ref{cha:content_blocks}.
    }
  \end{table}

  An example of a pattern match part might be the following.
  %
  {\small
  \begin{Verbatim}
    {
      type: patternmatch
      marks: 1
      prompt: Write the name of a Top Gear presenter.
      answer: (James( May)?)|(Richard( Hammond)?)|(Jeremy( Clarkson)?)
      displayanswer: Jeremy
      casesensitive: false
    }
  \end{Verbatim}
  }
  %
  In this example, the valid answers are \verb"James", \verb"Richard", \verb"Jeremy", \verb"James May", \verb"Richard Hammond", \verb"Jeremy Clarkson".  The matches to the correct answers are not case-sensitive.  Example invalid answers include \verb"JamesMay", \verb"Richard Clarkson", and \verb"Hammond Richard".

  When the correct answer is revealed, the name \verb"Jeremy" is shown.  It is
  up to you to define a \codeprop{displayanswer} which describes the possible
  correct answers.

  \textbf{Note:} How to write regular expressions is beyond the scope of this
  manual, but when using this part type, be absolutely sure that your regular
  expression is correct.  It is very easy to construct a regular expression
  which looks correct, but in fact, does not capture what you intend.

  \section{\label{sec:multiple_choice_part}Multiple choice}
  There are three part types which can be thought of as \emph{Multiple choice}.
  %
  \begin{description}
    \item[1\_n\_2] There are several answers, of which the student must pick
      one.  (For example,  a list of statements, only one of which is true.)
    \item[m\_n\_2] There are several answers, of which the student can pick any
      number. (For example, a list of statements, several of which are true.)
    \item[m\_n\_x] There is an $M \times N$ matrix of choice/answer pairs. The
      student will either have to select one cell from each row, or select any
      number of cells. (This type is equivalent to the previous two types, but
      repeated for multiple situations/objects.)
  \end{description}
  %
  In each case a marking matrix must be defined, which determines how many
  marks should be awarded for a particular choice.  For each choice that the
  student selects, the corresponding marks in the marking matrix are added to
  (or subtracted from) the score for the part.
  Table~\ref{tab:multiple_choice_part} details the available properties for
  this part type.
  %
  \begin{table}[ht]
    \centering
    \begin{tabular}{lp{20em}l}
      \hline
      Property & Description & Default value \\
      \hline
      \codepropreq{choices} & \emph{(Content)} An array of strings representing
      the available choices. & \nodef \\
      \codeprop{answers} & \emph{(Content)} If the part type is \verb"m_n_x",
      then an array of strings representing the possible answers to choose. &
      \verb"[]" \\
      \codepropreq{matrix} & The marking matrix.  An array of numbers giving
      the marks awarded for selecting each choice.  Negative numbers are
      allowed. If the part type is \verb"m_n_x", then the marking matrix must
      be an array of arrays, where each array corresponds to the marks for a
      row of choices.  & \nodef \\
      \codeprop{maxmarks} & The maximum number of marks which can be awarded
      for this part.  A value of 0 means no restriction. & \verb"0" \\
      \codeprop{maxanswers} & The maximum number of answers the student can
      select for each choice.  A value of 0 means no restriction. & \verb"0" \\
      \codeprop{minanswers} & The minimum number of answers the student can
      select for each choice. & \verb"0" \\
      \codeprop{shufflechoices} & Determines whether the choices should be
      displayed randomly. & \verb"false" \\
      \codeprop{shuffleanswers} & Determines whether the answers should be
      displayed randomly. & \verb"false" \\
      \codeprop{displaytype} & How to display the choices.  Possible values are
      \verb"radiogroup", \verb"dropdownlist", and \verb"checkbox". &
      \verb"radiogroup" \\
      \codeprop{displaycolumns} & Choices in the \verb"1_n_2" and \verb"m_n_2"
      part types are spread over this many columns, or displayed in a single
      column list if the value is 0. & \verb"0" \\
      \hline\hline
    \end{tabular}
    \caption{\label{tab:multiple_choice_part}
      The valid properties of a multiple choice part.  The \codeprop{choices}
      and \codeprop{matrix} properties are required, as denoted by the italic
      typeface.  The \codeprop{choices} and \codeprop{answers} properties are
      \codecontent{content} blocks --- see chapter~\ref{cha:content_blocks}.
    }
  \end{table}

  An example of an \verb"m_n_2" part type might be the following.
  %
  \begin{Verbatim}
    {
      type: m_n_2
      marks: 2
      prompt: Which of the following are true?
      choices: [
        "$\sqrt{4} = 2$"
        "$x^2 = x$"
        "$x^2+x+1 = 0$ has no real solutions"
      ]
      matrix: [1,-1,1]
      minmummarks: 0
      shufflechoices: true
      displaycolumns: 3
    }
  \end{Verbatim}
  %
  Here, we have chosen the \verb"m_n_2" part type, so the student must select a
  number of choices.  If the student chooses the first and third choices (as
  defined in the question), then they will be awarded one mark for each.
  Choosing the second choice will result in the student having one mark
  deducted.  The student will see the choices displayed in a random order, and
  the student's mark for this part cannot be negative, owing to the
  \codeprop{minmummarks} property being set to zero (the default).

  \section{\label{sec:gapfill_part}Gap fill}
  The \emph{Gap fill} part type presents a block of content to the student,
  with ``gaps'' which they must fill in.  Each gap is defined in the same way
  as any other part.  This part type makes it possible to combine several other
  part types in a single question part.  The following example will better
  illustrate how to use this part type.
  %
  \begin{Verbatim}
    {
      type: gapfill
      prompt: "
        The equation $x^2 + 3x + 2 = 0$ has [[0]] linear factors.

        The least solution is $x = $[[1]]

        The greatest solution is $x = $[[2]]
      "
      gaps: [
        { type: numberentry, answer: 2, marks: 1 }
        { type: jme, answer: x + 1, marks: 1 }
        { type: jme, answer: x + 2, marks: 1 }
      ]
    }
  \end{Verbatim}
  %
  The gap fill part type has only one property, namely \codeprop{gaps}, which
  is an array of \codeobject{part} objects.  In the example above there are
  three gaps.  The first is a number entry part, which corresponds to the
  placeholder given by \verb"[[0]]".  In other words, the student will be
  expected to enter an answer in the input box which will appear between the
  words \emph{has} and \emph{linear}.

  Similarly, input boxes will appear in place of \verb"[[1]]" and \verb"[[2]]",
  for which the student is expected to enter JME expressions.

  The total number of marks available for the part is calculated automatically
  from the marks available for each gap.

  \section{\label{sec:information_part}Information}
  The \emph{Information} part is not really a part type in the same sense as
  the others.  It is simply a block of content with no special properties,
  often used to give students hints in more difficult questions, \eg
  %
  \begin{Verbatim}
    {
      type: information
      prompt: "Pythagoras' Theorem states: $a^2 = \sqrt{b^2 + c^2}$"
    }
  \end{Verbatim}

\end{chapter}
