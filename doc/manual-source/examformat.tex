%Copyright 2011 Newcastle University
%
%   Licensed under the Apache License, Version 2.0 (the "License");
%   you may not use this file except in compliance with the License.
%   You may obtain a copy of the License at
%
%       http://www.apache.org/licenses/LICENSE-2.0
%
%   Unless required by applicable law or agreed to in writing, software
%   distributed under the License is distributed on an "AS IS" BASIS,
%   WITHOUT WARRANTIES OR CONDITIONS OF ANY KIND, either express or implied.
%   See the License for the specific language governing permissions and
%   limitations under the License.

\begin{chapter}{\label{cha:examformat}The exam format}
  As we briefly explained in chapter~\ref{cha:quickstart}, an exam is
  constructed by writing a plain text \codefile{.exam} file, which consists of
  simple markup, describing all aspects of the exam.  This chapter describes
  the fundamental aspects of the markup.

  \section{Data types}
  The markup is a JSON-like format (\url{http://www.json.org/}), with minimal
  punctuation, and three data types: objects, arrays, and literals. The
  language is purely declarative --- there is no control code.

  \subsection{Arrays}
  Arrays take the form
  %
  \begin{Verbatim}
    [ data1, data2, ..., dataN ]
  \end{Verbatim}
  %
  or, alternatively, across multiple lines
  %
  \begin{Verbatim}
    [
      data1
      data2
      ...
      dataN
    ]
  \end{Verbatim}
  %
  in which case commas can be omitted.

  \subsection{Objects}
  Objects are lists of key--value pairs, and take the form
  %
  \begin{Verbatim}
    { key1: value1, key2: value2, ..., keyN: valueN }
  \end{Verbatim}
  %
  where each value can be of any data type, including other objects.  Again, it
  is possible to write an object over multiple lines, optionally omitting the
  separating commas:
  %
  \begin{Verbatim}
    {
      key1: value1
      key2: value2
      key3: {
        key4: value4
        key5: value5
      }
      keyN: valueN
    }
  \end{Verbatim}
  
  \subsection{\label{sec:literals_quoting}Literals and quoting}
  A literal is a text string, which may or may not need to be quoted.  An
  unquoted literal ends with a new line or a comma, \eg
  %
  \begin{Verbatim}
    {
      name: My First exam, age: 25
    }
  \end{Verbatim}
  %
  Properties of exams or questions or part types, or anything which is not
  marked as content, will very rarely need to be quoted.  Pure number values,
  and boolean values also do not need to be quoted.  Some examples of when to
  use quotes are shown below.
  %
  \begin{description}
    \item[Single-quotes:] Used only when defining string variables (variables
      are explained in later chapters).  Consider defining a string variable
      \verb"hello" as below.  In this case, the value must be enclosed in
      single-quotes.
      %
      \begin{Verbatim}
    variables: {
      a: 'Hello'
    }
      \end{Verbatim}
      %
    \item[Double-quotes:] Used when the property value includes a new line or a
      comma.  Content which includes braces or square brackets should also be
      enclosed in double-quotes.  The value below includes both a new line and
      a comma, therefore double-quotes must be used.
      %
      \begin{Verbatim}
    {
      prompt1: "Hello, this is a
        long string"
    }
      \end{Verbatim}
      %
    \item[Triple double-quotes (\texttt{"""}):] Used when the property value
      itself includes double-quotes.  The value below itself includes
      a double-quote, so it must be enclosed in triple double-quotes.
      %
      \begin{Verbatim}
    {
      prompt2: """
       Bob said "Hello" to Fred
	  """
    }
      \end{Verbatim}
      %
  \end{description}

\end{chapter}
