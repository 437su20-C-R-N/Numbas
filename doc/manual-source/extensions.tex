%Copyright 2011 Newcastle University
%
%   Licensed under the Apache License, Version 2.0 (the "License");
%   you may not use this file except in compliance with the License.
%   You may obtain a copy of the License at
%
%       http://www.apache.org/licenses/LICENSE-2.0
%
%   Unless required by applicable law or agreed to in writing, software
%   distributed under the License is distributed on an "AS IS" BASIS,
%   WITHOUT WARRANTIES OR CONDITIONS OF ANY KIND, either express or implied.
%   See the License for the specific language governing permissions and
%   limitations under the License.

\begin{chapter}{\label{cha:extensions}Extensions}
  It has already been mentioned that it is possible to define new functions to 
  be used when writing an exam. When you unpack Numbas you will find that there
  are already some JavaScript files in the \codefile{extensions} directory. 
  
  \section{\label{sec:stats}Statistics Extension}
  Table~\ref{tab:statsfunctions1} details all statistical functions that have
  already been defined as part of the \codefile{stats.js} extension file, to
  generate random variables and calculate PMFs, PDFs and CDFs for various
  distributions. Table~\ref{tab:statsfunctions2} details all other statistical
  functions. 
  %
  \begin{sidewaystable}[ht]
  	\centering
  	\begin{tabular}{lllp{20em}}
  		\hline
  		Function & Argument type & Result type & Meaning \\
  		\hline
      \verb"randomBernoulli(p)" & number & number & Random Bernoulli variable, probability $p$ \\
      \verb"randomBinomial(n,p)" & number, number & number & Random Binomial variable, parameters $n$, $p$ \\
      \verb"pmfBinomial(x,n,p)" & number, number, number & number	& Binomial variable PMF, parameters $n$, $p$ \\
      \verb"randomUniform(a,b)" & number, number & number & Random Uniform variable, parameters $a$, $b$ \\
      \verb"pmfUniform(x,a,b)" & number, number, number & number & Uniform variable PMF, parameters $a$, $b$ \\
      \verb"cdfUniform(x,a,b)" & number, number, number & number & Uniform variable CDF, parameters $a$, $b$ \\
      \verb"randomPoisson(lambda)"	& number & number & Random Poisson variable, parameter $\lambda$ \\
      \verb"pmfPoisson(x,lambda)" & number, number & number & Poisson variable PMF, parameter $\lambda$ \\
      \verb"randomGeometric(p)" & number & number & Random Geometric variable, probability $p$ \\
      \verb"pmfGeometric(x,p)" & number, number & number & Geometric variable PMF, probabiltiy $p$ \\
      \verb"cdfGeometric(x,p)"	& number, number & number & Geometric variable CDF, probability $p$ \\
      \verb"randomExponential(lambda)" & number & number & Random Exponential variable, parameter $\lambda$ \\
      \verb"pdfExponential(x,lambda)" & number, number & number & Exponential distribution PDF, parameter $\lambda$ \\
      \verb"cdfExponential(x,lambda)" & number, number & number & Exponential distribution CDF, parameter $\lambda$ \\
      \verb"randomNormal(mu,sigma)" & number, number & number & Random Normal variable, mean $\mu$, standard deviation $\sigma$ \\
      \verb"pdfNormal(z,mu,sigma)" & number, number, number & number & Normal distribution PDF, mean $\mu$, standard deviation $\sigma$ \\
  		\verb"cdfNormal(z)" & number & number & Normal distribution CDF, mean $0$, variance $1$ \\
      \verb"randomGamma(n,lambda)" & number, number & number & Random Gamma variable, parameters $n$, $\lambda$ \\
      \verb"pdfGamma(x,n,lambda)" & number, number, number & number & Gamma distribution PDF, parameters $n$, $\lambda$ \\
      \hline\hline
  	\end{tabular}
  	\caption{\label{tab:statsfunctions1}
  		Functions to do with certain statistical distributions.
  	}
  \end{sidewaystable}
  %
  \begin{sidewaystable}[ht]
  	\centering
  	\begin{tabular}{lllp{20em}}
  		\hline
  		Function & Argument type & Result type & Meaning \\
  		\hline
      \verb"mean(data)"	& list & number	& Mean of a list of values \\
      \verb"variance(data)" & list & number & Variance of a list of values \\
      \verb"standardDev(data)" & list & number & Standard deviation of a list of values \\
      \verb"regression(data1,data2,z)" & list, list, number & number & Uses \emph{Method of Least Squares} to find a line of best fit. 
      $z = 1$ returns intersect, $z = 2$ returns gradient. \\ 
      \verb"zTest(data,mu,v)" & list, number, number & number & Z-test for a sample of variables from a Normal distribution with mean $\mu$ and variance $v$.\\
      \hline\hline
  	\end{tabular}
  	\caption{\label{tab:statsfunctions2}
  		Statistical functions.
  	}
  \end{sidewaystable}
  %
  The functions generating certain types of random variables would most likely
  be used inside a \verb"variables" object. For example, the following defines
  $X$ as a Bernoulli random variable with success probability $p$.
  %
  \begin{Verbatim}
    variables: {
      X: "randomBernoulli(p)"
    }
  \end{Verbatim}
  %
  There is also a command to perform linear regression. However, we must output
  the gradient and intersect of the line of best fit individually.  The
  following prints the line of best fit for the data set $(1,3), (3,12),
  (7,23), (19,34), (36,40)$.
  %
  \begin{Verbatim}
    variables: {
      x: [1,3,7,19,36]
      y: [3,12,23,34,40]
      a: "precround(regression(x,y,1),3)"
      b: "precround(regression(x,y,2),3)"
    }
    
    statement: """
      The line of best fit is $y = \var{a} + \var{b}x$.
    """  
	\end{Verbatim}
	%
  Further examples can be found in the \codefile{stats-extension.exam}, located
  in the \codefile{exams} directory.
		
\end{chapter}
