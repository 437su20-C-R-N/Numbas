%Copyright 2011 Newcastle University
%
%   Licensed under the Apache License, Version 2.0 (the "License");
%   you may not use this file except in compliance with the License.
%   You may obtain a copy of the License at
%
%       http://www.apache.org/licenses/LICENSE-2.0
%
%   Unless required by applicable law or agreed to in writing, software
%   distributed under the License is distributed on an "AS IS" BASIS,
%   WITHOUT WARRANTIES OR CONDITIONS OF ANY KIND, either express or implied.
%   See the License for the specific language governing permissions and
%   limitations under the License.

\begin{chapter}{\label{cha:exam_object}The exam object}
  The fundamental data structure is the \codeobject{exam} object, containing
  all the data necessary to define an exam.  This chapter describes the \codeobject{exam} object, and the data types and objects which can be used
  within it.

  By default, all object properties are optional, and if they are not present
  in the exam file, they will take their default values.
  Table~\ref{tab:exam_object} describes the properties of the exam itself.
  %
  \begin{table}[ht]
    \centering
    \begin{tabular}{lp{18em}l}
      \hline
      Property & Description & Default value \\
      \hline
      \codeprop{name} & The name of the exam as it appears at the top of the
      page. & \verb"Untitled exam" \\
      \codeprop{duration} & The time allowed for the exam, in seconds.  A value
      of 0 means there is no time limit. & \verb"0" \\
      \codeprop{percentpass} & The minimum percentage score to be classified as
      a pass. & \verb"0" \\
      \codeprop{shufflequestions} & Determines whether the question order
      should be randomised. & \verb"false" \\
      \codeprop{navigation} & An object defining navigation rules, \ie how the
      student is allowed to move between questions (see
      \S\ref{sec:navigation}). & \verb"{}" \\
      \codeprop{timing} & An object defining the warning messages shown to the
      student when they have run out of time, or when a certain amount of time
      is left (see \S\ref{sec:timing}). & \verb"{}" \\
      \codeprop{feedback} & An object defining feedback rules (see
      \S\ref{sec:feedback}). & \verb"{}" \\
      \codeprop{resources} & An array of directories or file names (relative to
      the directory containing the exam) to be included in the resources
      directory of a compiled exam (see \S\ref{sec:resources}). & \verb"[]" \\
      \codepropreq{questions} & An array of \codeobject{question} objects (see
      chapter~\ref{cha:question_object}). & \nodef \\
      \hline\hline
    \end{tabular}
    \caption{\label{tab:exam_object}
    The valid properties of an exam.  The \codeprop{questions} property is
    required, as denoted by the italic typeface.
    }
  \end{table}

  \section{\label{sec:event_object}Event object}
  One of the properties of an object might be an \codeobject{event} object.
  The \codeobject{event} object defines what action to take when a particular
  event occurs.  Table~\ref{tab:event_object} describes the properties of the
  \codeobject{event} object.
  %
  \begin{table}[ht]
    \centering
    \begin{tabular}{lp{20em}l}
      \hline
      Property & Description & Default value \\
      \hline
      \codeprop{action} & What to do when a particular event occurs (see
      below).  & \verb"None" \\
      \codeprop{message} & The message to display when an event occurs,
      providing \codeprop{action} is not \verb"None". & \emstr \\
      \hline\hline
    \end{tabular}
    \caption{\label{tab:event_object}
      The valid properties of an event object.
    }
  \end{table}

  The \codeprop{action} property can take various values, depending on the
  context under which the \codeobject{event} object appears.  The permissible
  values are described elsewhere, where an \codeobject{event} object is a
  property of another object.

  \section{\label{sec:navigation}Navigation}
  It is possible to control how and when students are allowed to move between
  questions, for example, whether they are allowed to return to a question once
  they have completed it, whether they can jump between questions at will, or
  whether they must complete a question before moving on to another one.

  Navigation is an \codeobject{exam} property, and is controlled by the
  \codeobject{navigation} object.  A simple example might be the following:
  %
  \begin{Verbatim}
    navigation: {
      reverse: false
      browse: false
    }
  \end{Verbatim}
  %
  which means that the student is not allowed to return to a previous question
  (\codeprop{reverse}), and is not allowed to jump between questions at will
  (\codeprop{browse}), so the student can only move forward through the exam.
  By default, both of these properties are \verb"true".

  Table~\ref{tab:navigation_object} describes the properties of the
  \codeobject{navigation} object.
  %
  \begin{table}[ht]
    \centering
    \begin{tabular}{lp{20em}l}
      \hline
      Property & Description & Default value \\
      \hline
      \codeprop{reverse} & Whether the student is allowed to move to the
      previous question. & \verb"true" \\
      \codeprop{browse} & Whether the student is allowed to move to any
      question, by using the question list. & \verb"true" \\
      \codeprop{onadvance} & An \codeobject{event} object describing what to do
      when a student attempts to move on to the next question, without
      completing the current question. &
      \mbox{see \S\ref{sec:nav_events}} \\
      \codeprop{onreverse} & An \codeobject{event} object describing what to do
      when a student attempts to move to the previous question, without
      completing the current question. &
      \mbox{see \S\ref{sec:nav_events}} \\
      \codeprop{onmove} & An \codeobject{event} object describing what to do
      when a student attempts to move on to any other question, without
      completing the current question. &
      \mbox{see \S\ref{sec:nav_events}} \\
      \hline\hline
    \end{tabular}
    \caption{\label{tab:navigation_object}
      The valid properties of a navigation object.
    }
  \end{table}

  \subsection{\label{sec:nav_events}Navigation events}
  The \codeprop{onadvance}, \codeprop{onreverse}, and \codeprop{onmove}
  properties are \codeobject{event} objects (see \S\ref{sec:event_object}),
  which determine what action should be taken when the student attempts to
  navigate away from the current question, having not completed the
  current question.  They take the default value of
  %
  \begin{Verbatim}
    { action: none, message: "" }
  \end{Verbatim}
  %
  so the student is allowed to move back or forward to another question, or
  jump between questions in the question list unhindered.

  The permissible values for the \codeprop{action} property of the
  \codeobject{event} objects are:
  %
  \begin{itemize}
    \item \verb"none": do nothing;
    \item \verb"warnifunattempted": warn the student that they have not
      completed the current question, but move on anyway;
    \item \verb"preventifunattempted": warn the student that they have not
      completed the current question, and do not allow the student to move on
      to the next question.
  \end{itemize}

  \section{\label{sec:timing}Timing}
  If the assignment is timed, by setting the exam property \codeprop{duration}
  to a non-zero value, then the \codeobject{timing} object can be used to
  define warning messages shown to the student when a certain amount of time is
  left, or when the time has expired.  Table~\ref{tab:timing_object} describes
  the valid properties of the \codeobject{timing} object.
  %
  \begin{table}[ht]
    \centering
    \begin{tabular}{lp{20em}l}
      \hline
      Property & Description & Default value \\
      \hline
      \codeprop{timeout} & An \codeobject{event} object describing what to do
      when the student has run out of time. & see
      \mbox{\S\ref{sec:timing_events}} \\
      \codeprop{timedwarning} & An \codeobject{event} object describing what to
      do when the student has five minutes left. & see
      \mbox{\S\ref{sec:timing_events}} \\
      \hline\hline
    \end{tabular}
    \caption{\label{tab:timing_object}
      The valid properties of a timing object.
    }
  \end{table}

  \subsection{\label{sec:timing_events}Timing events}
  The \codeprop{timeout} and \codeprop{timedwarning} properties are
  \codeprop{event} objects (see \S\ref{sec:event_object}), which determine what
  action to take when a student has run out of time, or when there is a certain
  amount of time left (this is fixed at five minutes at the moment).  The
  default for both properties is
  %
  \begin{Verbatim}
    {
      action: none
      message: ""
    }
  \end{Verbatim}
  %
  so no action is taken in either case.  The valid values for the
  \codeprop{action} property are
  %
  \begin{itemize}
    \item \verb"none": do nothing;
    \item \verb"warn": display the warning \codeprop{message}.
  \end{itemize}

  \section{\label{sec:feedback}Feedback and advice}
  \numbas\ includes the facility to provide feedback and advice to the
  student taking the exam.  The most common use of feedback is to provide the
  student with a fully worked solution to a question.  The student can see this
  feedback by pressing the \codebutton{Reveal} button (see the screen shot in
  figure~\ref{fig:example_screen2}).  You can control when and how the student
  sees this feedback, by using the \codeobject{feedback} object, which is an
  \codeobject{exam} property.  All valid properties of the
  \codeobject{feedback} object are detailed in table~\ref{tab:feedback_object}.
  %
  \begin{table}[ht]
    \centering
    \begin{tabular}{lp{18em}l}
      \hline
      Property & Description & Default value \\
      \hline
      \codeprop{showactualmark} & Whether to show the student's score while the
      exam is in progress. & \verb"true" \\
      \codeprop{showtotalmark} & Whether to show the total marks available for
      the exam, questions, and question parts. & \verb"true" \\
      \codeprop{showanswerstate} & Whether to show the tick, cross, or percent
      sign when a student submits an answer. & \verb"true" \\
      \codeprop{allowrevealanswer} & Whether the student is allowed to click
      the \codebutton{Reveal} button to see the advice. & \verb"true" \\
      \codeprop{advice} & An \codeobject{advice} object defining under which
      circumstances advice is shown (see \S\ref{sec:advice_object}). &
      \verb"{}" \\
      \hline\hline
    \end{tabular}
    \caption{\label{tab:feedback_object}
      The valid properties of a feedback object.
    }
  \end{table}
  
  \subsection{Restricting advice}
  By default, the student is allowed to reveal the advice to a question at any
  time, by pressing the \codebutton{Reveal} button.  Doing so will remove all
  marks awarded for the current question (the student is warned that this will
  happen).  To prevent the student being able to see the advice, set the
  \codeprop{allowrevealanswer} property to \verb"false", \eg
  %
  \begin{Verbatim}
    feedback: {
      allowrevealanswer: false
    }
  \end{Verbatim}
  %
  This will remove the \codebutton{Reveal} button from the page.

  \subsection{Restricting mark information}
  By default, the student can see how many marks are available for the entire
  exam, and how many marks have been awarded so far.  This behaviour can be
  altered by setting the \codeprop{showtotalmark} and \codeprop{showactualmark}
  properties, \eg
  %
  \begin{Verbatim}
    feedback: {
      showtotalmark: false
      showactualmark: false
    }
  \end{Verbatim}
  %
  This will prevent the student from being able to see how many marks have been
  awarded, and how many are available for the exam.

  \subsection{Restricting the answer state}
  By default, the student will be shown whether the submitted answer is correct
  (green tick), incorrect (red cross), or --- in the case of multiple submit
  boxes --- partially correct (blue percent sign).

  You can alter this behaviour with the \codeprop{showanswerstate} property,
  \eg
  %
  \begin{Verbatim}
    feedback: {
      showanswerstate: false
    }
  \end{Verbatim}
  %
  In this case, the student receives no indication of whether the submitted
  answer is correct.

  \textbf{Note:} Be careful of how you use \codeprop{showanswerstate} and
  \codeprop{showactualmark}.  If you set the former to \verb"false", but the
  latter to \verb"true", then students can still see whether their answers are
  correct by looking at the current mark total.  In contrast, it is reasonable
  to set \codeprop{showanswerstate} to \verb"true", but
  \codeprop{showactualmark} to \verb"false", when you want to show whether the
  answer is correct, but not how many marks are gained.

  \subsection{\label{sec:advice_object}Advice object}
  The \codeobject{advice} object defines under which circumstances advice is
  shown to the student.  Advice is a property of the \codeobject{feedback}
  object.  Table~\ref{tab:advice_object} describes the properties
  of the \codeobject{advice} object.
  %
  \begin{table}[ht]
    \centering
    \begin{tabular}{lp{20em}l}
      \hline
      Property & Description & Default value \\
      \hline
      \codeprop{type} & When advice is shown to the student.  If the value is
      \verb"onreveal", then advice is only shown to the student when the
      \codebutton{Reveal} button is pressed.  If the value is \verb"threshold",
      then advice is shown when the student scores below
      \codeprop{threshold}\%.  & \verb"onreveal" \\
      \codeprop{threshold} & Reveal advice if the student scores less than this
      percentage on a single question. & \verb"0" \\
      \hline\hline
    \end{tabular}
    \caption{\label{tab:advice_object}
      The valid properties of an advice object.
    }
  \end{table}

  As a property of the \codeobject{feedback} object, the \codeobject{advice}
  object determines when advice is shown to the student.  The actual
  advice content shown to the student is set with the \codeprop{advice}
  property of the \codeobject{question} object --- see
  \S\ref{sec:advice_content}.

  \subsection{Feedback examples}
  There are a number of feedback and advice options, and choosing the correct
  ones for the behaviour you desire can be tricky at first.  Some examples are
  listed below.

  \subsubsection{All feedback}
  Set no feedback options explicitly; the default options will turn on all
  forms of feedback except threshold advice.

  \subsubsection{No feedback}
  Use a \codeobject{feedback} object with the following properties set to 
	\verb"false".
  %
  \begin{Verbatim}
    feedback: {
      showtotalmark: false
      showactualmark: false
      showanswerstate: false
      allowrevealanswer: false
    }
  \end{Verbatim}

  \subsubsection{No Reveal button but automatic feedback when student scores
    less than a threshold}
  Use a \codeobject{feedback} object with \codeprop{showanswerstate} set to
  \verb"false".  Also use the feedback \codeobject{advice} object with
  \codeprop{type} set to \verb"threshold", and \codeprop{threshold} set to the
  threshold percentage.
   %
  \begin{Verbatim}
    feedback: {
      allowrevealanswer: false
      advice: {
        type: threshold
        threshold: n
      }
    }
  \end{Verbatim}

  \subsubsection{Reveal button and automatic feedback when student scores less
    than a threshold}
  Use a \codeobject{feedback} object with an \codeobject{advice} object.  Set
  \codeprop{type} to \verb"threshold", and \codeprop{threshold} to the
  threshold percentage.
  %
  \begin{Verbatim}
    feedback: {
      advice: {
        type: threshold
        threshold: n
      }
    }
  \end{Verbatim}

  \section{\label{sec:resources}Resources}
  If you want to include videos or images in your exams, then they must be part
  of the collection of files which make up the compiled exam.

  This can be achieved by setting the \codeprop{resources} property of the
  \codeobject{exam} object to an array of directories or file names (relative
  to the directory in which the \codefile{.exam} file resides), which are to be
  included.  These directories and files then appear under the
  \codefile{resources} directory of the compiled exam.

  As an example, suppose you want to include the images \verb"graph.png" and
  \verb"ball.png" in an exam.  You should set \codeprop{resources} as
  %
  \begin{Verbatim}
    resources: [graph.png, ball.png]
  \end{Verbatim}
  %
  Then, when you need to make reference to \verb"graph.png" in a question, for
  example, the path to the image is \verb"resources/graph.png".  Images and
  videos can be included in \codecontent{content} blocks, which are described
  later in chapter~\ref{cha:content_blocks}.

\end{chapter}
