%Copyright 2011 Newcastle University
%
%   Licensed under the Apache License, Version 2.0 (the "License");
%   you may not use this file except in compliance with the License.
%   You may obtain a copy of the License at
%
%       http://www.apache.org/licenses/LICENSE-2.0
%
%   Unless required by applicable law or agreed to in writing, software
%   distributed under the License is distributed on an "AS IS" BASIS,
%   WITHOUT WARRANTIES OR CONDITIONS OF ANY KIND, either express or implied.
%   See the License for the specific language governing permissions and
%   limitations under the License.

\begin{chapter}{\label{cha:part_object}The part object}
  The \codeobject{part} object defines the parts that make up a question.  It
  is a property of the \codeobject{question} object.  In addition to the
  properties listed in table~\ref{tab:part_object}, further properties are
  required, depending on the \emph{type} of the part.
  Chapter~\ref{cha:question_parts} describes the supported part types, and the
  additional properties, in detail.
  %
  \begin{table}[ht]
    \centering
    \begin{tabular}{lp{18em}l}
      \hline
      Property & Description & Default value \\
      \hline
      \codepropreq{type} & The part type.  Permissible values are \verb"jme",
      \verb"numberentry", \verb"patternmatch", \verb"1_n_2", \verb"m_n_2",
      \verb"m_n_x", \verb"gapfill", and \verb"information".  Part types are
      described in detail in chapter~\ref{cha:question_parts}. & \nodef \\
      \codeprop{marks} & The number of marks available for this part --- see
      \S\ref{sec:part_marks}. &
      \verb"0" \\
      \codeprop{minimummarks} & The minimum number of marks available for this
      part --- see \S\ref{sec:part_marks}. & \verb"0" \\
      \codeprop{enableminimummarks} & Whether there should be a minimum number
      of marks available for this part.  Set in conjunction with
      \codeprop{minimummarks} --- see \S\ref{sec:part_marks}. & \verb"false" \\
      \codeprop{prompt} & \emph{(Content)} Text telling the student what they
      should do. & \emstr \\
      \codeprop{steps} & An array of \codeobject{part} objects, which the
      student can reveal.  These parts can be used as intermediate steps in
      answering the question.  You can decide whether to deduct marks if the
      student uses steps, by setting \codeprop{stepspenalty} below.  See
      \S\ref{sec:steps} for more details on steps. & \verb"[]"
      \\
      \codeprop{stepspenalty} & The number of marks to deduct from the total
      available for this part, if the student uses steps. & \verb"0" \\
      \hline\hline
    \end{tabular}
    \caption{\label{tab:part_object}
    The valid properties of a part.  The \codeprop{type} property is required,
    as denoted by the italic typeface.  The \codeprop{prompt} property is a
    \codecontent{content} block --- see chapter~\ref{cha:content_blocks}.
    }
  \end{table}

  \section{\label{sec:part_marks}Part marks}
  There are three properties controlling the number of marks available for a
  part.  The first is \codeprop{marks}, which sets the maximum number of
  marks available for the part.  The \codeprop{minimummarks} and
  \codeprop{enableminimummarks} properties are most useful when you have
  defined part steps --- see \S\ref{sec:steps}.

  Since it is possible for the student to lose marks by revealing steps, it is
  possible for the part to be negatively marked.  For example, suppose there
  were 3 marks available for a part, and you decided that the student should
  lose 1 mark if they viewed steps.  If the student did view the steps, and
  also answered the part incorrectly, then their net score for this part would
  be $-1$.  To prevent this, you can set \codeprop{minimummark} to some value,
  zero say, and also set \codeprop{enableminimummark} to \verb"true".

  \section{\label{sec:steps}Steps}
  Steps can be used as ``intermediate'' question parts, where you would like to
  provide the student with the option of taking extra steps to answer a
  question.  Sometimes, the step may simply be informative, reminding the
  student of a formula, for example.  Alternatively, it might be a series of
  sub-parts guiding the student to the final answer.  Initially, the steps are
  hidden from the student.  Clicking on the \codebutton{Show steps} button will
  reveal the steps.  You can also decide whether marks should be deducted if
  the student views the steps.

  \subsection{Creating steps}
  You define steps using the \codeprop{steps} property of the \codeobject{part}
  object.  The \codeprop{steps} property is itself an array of
  \codeobject{part} objects, and so creating a step is exactly equivalent to
  creating a question part.  All the part types explained in
  chapter~\ref{cha:question_parts} are available to use.

  If you decide that marks should be deducted when the student views the
  steps, you can do so by setting the \codeprop{stepspenalty} property to a
  non-zero value.  This number is then subtracted from the maximum number of
  marks available for the part.  Note that this procedure can result in
  negative marking for parts --- see \S\ref{sec:part_marks}.
  
\end{chapter}
