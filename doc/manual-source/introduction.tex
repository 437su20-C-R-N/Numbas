%Copyright 2011 Newcastle University
%
%   Licensed under the Apache License, Version 2.0 (the "License");
%   you may not use this file except in compliance with the License.
%   You may obtain a copy of the License at
%
%       http://www.apache.org/licenses/LICENSE-2.0
%
%   Unless required by applicable law or agreed to in writing, software
%   distributed under the License is distributed on an "AS IS" BASIS,
%   WITHOUT WARRANTIES OR CONDITIONS OF ANY KIND, either express or implied.
%   See the License for the specific language governing permissions and
%   limitations under the License.

\begin{chapter}{\label{cha:introduction}Introduction}
  \numbas\ is a web-based e-assessment system developed at Newcastle
  University.  It consists of a set of tools which produce SCORM-compliant exam
  packages (\url{http://scorm.com/}).

  \section{Features}
  \begin{itemize}
    \item Simple installation.  Unpack \numbas\ anywhere on your computer.
    \item Web-based, so it can run on a wide range of computers.  In
      particular, Windows, Mac and Linux operating systems are supported, under
      Internet Explorer 8+, Mozilla Firefox 2.5+, Google Chrome and Safari.
    \item Runs entirely on the client computer in JavaScript.  This means there
      is no backend server, and exams can be deployed in a variety of
      locations, for example, in Virtual Learning Environments (VLEs) or
      Learning Management Systems (LMS) such as Blackboard, DVDs, and even
      stand-alone on the web.
    \item Extensive support for questions of a mathematical nature. Answers to
      questions can be complicated mathematical expressions.
    \item Support for \LaTeX, using MathJax (\url{http://www.mathjax.org/}).
    \item Write questions using simple markup with any text editor.
    \item Questions can be fully randomised.
    \item Because the exam runs in the browser, rich content such as videos and
      interactive graphs can be included.
    \item SCORM 2004 standards compliant, so the exams can be run in a VLE
      which supports this standard.
    \item Support for themes to change the look and user interface of exams.
    \item Support for extensions to add new features, such as new question
      types, or mathematical and statistical libraries.
  \end{itemize}

  \section{System requirements}
  \numbas\ has modest system requirements, whether you are running an
  exam or if you want to author questions.
  %
  \begin{itemize}
    \item A web browser on Windows, Linux, or Mac.  \numbas\ has been tested
      and is known to work with Internet Explorer (versions 8 and higher),
      Mozilla Firefox (versions 2.5 and higher), Google Chrome, and Safari.
    \item If you wish to author questions, you will need a text editor, and an
      installation of the Python programming language
      (\url{http://www.python.org/}), version 3.1 or higher.
  \end{itemize}
  %
  \subsection{\label{sec:mathjax_fonts}Important information about MathJax}
  \numbas\ uses MathJax to render mathematics using native web fonts, which are
  supported by all modern browsers.  Due to a security setting in Mozilla
  Firefox, web fonts cannot be loaded by the exam when it is run locally within
  the browser (\ie not from a web server), and MathJax falls back to rendering
  mathematics using images, which is slower.  (The layout of content on the
  page is not affected.)  This can be remedied, however, by installing a local
  copy of MathJax.  See appendix~\ref{cha:scorm} on SCORM (specifically
  \S\ref{sec:local_mathjax}).

  \section{\numbas\ structure}
  \numbas\ can be unpacked anywhere on your system, and does not require
  any further installation.  Once unpacked, the directory structure is as follows.
  %
  \begin{description}
    \item[bin:] Python scripts for compiling an exam.
    \item[doc:] The documentation for \numbas.
    \item[exams:] Files which describe your exams.
    \item[extensions:] You can extend the functionality of \numbas\ by writing
      JavaScript files and putting them in this directory.
    \item[output:] Once an exam is compiled, the output goes in this directory.
    \item[runtime:] The core JavaScript files which must be included with every \numbas\ exam.
    \item[scormfiles:] Additional files necessary to create a SCORM-compliant
      exam --- see appendix~\ref{cha:scorm}.
    \item[themes:] The look and feel of an exam can be
      customised by creating themes. Themes should go in this directory.
  \end{description}

  \section{Organisation of this manual}
  The manual is designed to be read from beginning to end.  Each chapter builds
  on the last.  Chapters should not be skipped, otherwise certain concepts are
  unlikely to make sense.

  The next chapter --- chapter~\ref{cha:quickstart} --- is intended as a quick
  start guide, by the end of which you should have a fully functioning exam,
  using the simple example exam provided.  Subsequent chapters explain \numbas\
  in detail.
  \begin{itemize}
    \item Chapter~\ref{cha:examformat} describes the basics of the markup used
      to construct exams.
    \item Chapter~\ref{cha:exam_object} describes the \codeobject{exam} object,
      which is used to define all aspects of an exam.
    \item Chapter~\ref{cha:question_object} describes the \codeobject{question}
      object, which is used to define questions with an exam.
    \item Chapter~\ref{cha:part_object} describes the \codeobject{part} object.
      Each question in your exam must have a part; this object is used to
      define the properties of a question part.
    \item Chapter~\ref{cha:content_blocks} describes content blocks.  These are
      used for prompting students to do something, and for the display of
      various aspects of the exam.
    \item Chapter~\ref{cha:question_parts} describes the part types which can
      be used within a question.
    \item Chapter~\ref{cha:jme_syntax} describes the Judged Mathematical
      Expression syntax, which is used to define variables, functions, and the
      answers to the JME part type.
    \item Appendix~\ref{cha:scorm} explains how to produce SCORM objects, which
      could be included in a VLE, such as Blackboard.
  \end{itemize}
\end{chapter}
