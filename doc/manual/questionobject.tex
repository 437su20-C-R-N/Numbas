%Copyright 2011 Newcastle University
%
%   Licensed under the Apache License, Version 2.0 (the "License");
%   you may not use this file except in compliance with the License.
%   You may obtain a copy of the License at
%
%       http://www.apache.org/licenses/LICENSE-2.0
%
%   Unless required by applicable law or agreed to in writing, software
%   distributed under the License is distributed on an "AS IS" BASIS,
%   WITHOUT WARRANTIES OR CONDITIONS OF ANY KIND, either express or implied.
%   See the License for the specific language governing permissions and
%   limitations under the License.

\begin{chapter}{\label{cha:question_object}The question object}
  The \codeobject{question} object defines the questions that make up an exam,
  and is a property of the \codeobject{exam} object.  Each question in an exam
  is made up of a number of parts, defined by the \codeobject{part} object,
  which is described in detail in chapter~\ref{cha:part_object}.  Each
  \codeobject{question} object can have any or all of the properties listed in
  table~\ref{tab:question_object}.
  %
  \begin{table}[ht]
    \centering
    \begin{tabular}{lp{18em}l}
      \hline
      Property & Description & Default value \\
      \hline
      \codeprop{name} & Name of the question. & \verb"Untitled question" \\
      \codeprop{statement} & \emph{(Content)} Question statement. Displayed
      above the question parts.  Tells the student what the setup of the
      question is. & \emstr \\
      \codeprop{advice} & \emph{(Content)} The text to display when the student
      presses the \codebutton{Reveal} button, or enters an incorrect answer ---
      see \S\ref{sec:advice_content}. &
      \emstr \\
      \codeprop{parts} & An array of \codeobject{part} objects (see
      chapter~\ref{cha:part_object}). & \verb"[]" \\
      \codeprop{variables} & An object defining the variables used in the
      question --- see \S\ref{sec:declaring_variables}. & \verb"{}" \\
      \codeprop{functions} & An object defining the functions used in the
      question --- see \S\ref{sec:functions}. & \verb"{}" \\
      \hline\hline
    \end{tabular}
    \caption{\label{tab:question_object}
    The valid properties of a question.  The \codeprop{statement} and
    \codeprop{advice} properties are \codecontent{content} blocks.  See
    chapter~\ref{cha:content_blocks} for more information on
    \codecontent{content} blocks.
    }
  \end{table}

  \section{Question parts}
  Each question in an exam has a question statement, which tells the student
  the setup of the question, and a number of question parts, which are defined
  in the \codeprop{parts} property.  The \codeprop{parts} property is an array
  of \codeobject{part} objects, described in detail in
  chapter~\ref{cha:part_object}.

  \section{\label{sec:declaring_variables}Declaring variables}
  Question variables are declared in a \codeobject{variables} object, which is
  a property of the \codeobject{question} object.  They can therefore be used
  in all question parts contained within the \codeobject{question} object.
  Each variable takes the key--value form \codevar{name: expr}, where
  \codevar{name} is the name of the variable, and \codevar{expr} is an
  expression defining the variable.  The expression will sometimes needed to be
  quoted, as explained in \S\ref{sec:literals_quoting}.
  
  The expression \codevar{expr} uses a special \emph{Judged Mathematical
  Expression} (JME) syntax.  Here, we shall provide a brief explanation of this
  syntax.  It is explained in greater depth in chapter~\ref{cha:jme_syntax}.

  \subsection{Simple variables}
  The value of a variable can be something as simple as a number, \eg
  %
  \begin{Verbatim}
    variables: {
      x: 5
      y: 3.7
    }
  \end{Verbatim}
  %
  It is also possible to reference other variables which have already been
  defined within the question, \eg
  %
  \begin{Verbatim}
    variables: {
      x: 5
      y: 3.7
      a: x+y
      b: a*x-7.4
      c: x/2+b^2
    }
  \end{Verbatim}
  %
  where we have also used simple mathematical operators (\verb"+", \verb"-",
  \verb"*", \verb"/", \verb"^") to combine variables.  Note that variable
  definitions cannot be cyclical, \eg \verb"a: b+2", \verb"b: a-2" is not
  permitted.

  Variables are not limited to being numerical; they can also be strings, \eg
  %
  \begin{Verbatim}
    variables: {
      surname: 'Bloggs'
      forename: 'Joe'
      name: forename+" "+surname
    }
  \end{Verbatim}
  %
  A string variable must be enclosed in single-quotes (double-quotes should not
  be used).  To concatenate strings use the \verb"+" operator, as in
  \codevar{name} above.

  \textbf{Note:} Variables cannot be named \codevar{pi}, \codevar{e}, or
  \codevar{i}, since these names are reserved for the built-in constants
  $\pi=3.14\ldots$, $\mathrm{e}=2.71\ldots$, and imaginary $i$.

  \subsection{Complex variables}
  \numbas\ supports complex numbers, and complex variables can be defined and
  used as you might expect, \eg
  \begin{Verbatim}
    variables: {
      z1: 3+5*i
      z2: 1-7*i
      z3: z1^2
      z4: z1-3*z2
      z5: re(z1)
      z6: im(z3)
    }
  \end{Verbatim}
  %
  The variables \codevar{z5} and \codevar{z6} make use of the functions
  \verb"re()", and \verb"im()", which return the real and imaginary parts of a
  complex number, respectively.  Many of the other available functions can also
  be applied to complex numbers.  See \S\ref{sec:jme_op_func} for a full list
  of valid mathematical operators, constructs, and functions.

  \subsection{\label{sec:randomisation}Randomisation}
  The variables in the previous section are static, \ie whenever the exam is
  run, these variables will always take the same values.  Much more flexibility
  can be gained by defining random variables, so that values change each time
  the exam is run.

  The syntax for a random variable is either
  %
  \begin{Verbatim}
    name: random(start..end[#step])
  \end{Verbatim}
  %
  or
  %
  \begin{Verbatim}
    name: "random(a1,a2,a3,...,aN)"
  \end{Verbatim}
  %
  The syntax \verb"start..end#step" means the variable will be chosen
  discretely from between \verb"start" and \verb"end" in steps of size
  \verb"step" (and \verb"step" is optional).  If the step size is zero, then
  the variable will be chosen continuously from the range.  If the step size is
  not set, then it defaults to one.  The syntax \verb"a1,a2,...,aN" means the
  variable will take one of the values \verb"a1" to \verb"aN".  Random
  variables of any data type (\eg number, string, boolean) can be defined.  The
  values \verb"a1,a2,...,aN" can themselves be variables that have already been
  properly assigned.  Some randomisation examples are shown in
  table~\ref{tab:randomisation}.
  %
  \begin{table}[ht]
    \centering
    \begin{tabular}{lp{16em}}
      \hline
      Randomisation                  & Possible values \\
      \hline
      \verb'v1: random(1..5)'        & 1, 2, 3, 4, 5 \\
      \verb'v2: random(10..20#2)'    & 10, 12, 14, 16, 18, 20 \\
      \verb'v3: random(0..0.5#0.05)' & 0, 0.05, 0.1, 0.15, \ldots, 0.5 \\
      \verb'v4: random(0..0.5#0)'    & Any real number in the range $[0,0.5]$ \\
      \verb'v5: "random(1,3,6,24)"'  & 1, 3, 6, 24 \\
      \verb'v6: random(-9..9)'       & -9, -8, \ldots, 8, 9 \\
      \verb'v7: "random(v1,v2,v3)"'  & The assigned value of either \verb"v1",
      \verb"v2", or \verb"v3" \\
      \verb+v8: "random('Cat','Dog','Bird')"+ & ``Cat'', ``Dog'', or ``Bird''
      \\
      \hline\hline
    \end{tabular}
    \caption{\label{tab:randomisation}
      Example random variables.
    }
  \end{table}
  %
  As with the simple variable declarations in the previous section, it is
  possible to reference other random variables which have already been
  declared, e.g. \verb"v9: 2*v1+random(3..7)", \verb'v10: "v2*random(2,4,12)/2"'.

  \subsection{\label{sec:conditionals}Conditional expressions}
  It is possible to declare variables conditionally, \ie a variable is assigned
  a value depending on whether a condition is true or false.  \numbas\
  supports \verb"if" statements and \verb"switch" statements.
  
  \subsubsection{if statements}
  The syntax for an \verb"if" statement is
  %
  \begin{Verbatim}
    name: "if(condition, expr1, expr2)"
  \end{Verbatim}
  %
  So, if \verb"condition" is \verb"true", then \verb"name=expr1", otherwise,
  \verb"name=expr2".  More complex conditions can be built up by using the
  \verb"and", \verb"or", and \verb"not" operators.  The expressions can include
  any valid JME syntax, including further \verb"if" or \verb"switch" statements
  (see below).  Some examples are shown in table~\ref{tab:if_statements}.
  %
  \begin{table}[ht]
    \centering
    \begin{tabular}{ll}
      \hline
      Variable         & Value \\
      \hline
      \codevar{amount} & \verb"random(5000..15000#1000)" \\
      \codevar{rate}   & \verb'"if(amount<10000,0.1,0.15)"' \\
      \codevar{tax}    & \verb"amount*rate" \\
      \hline
      \codevar{x}      & \verb"random(1..5)" \\
      \codevar{y}      & \verb"random(2..10)" \\
      \codevar{z}      & \verb'"if(x<3 and y>7,random(-5..-1),x*y^2)"' \\
      \hline\hline
    \end{tabular}
    \caption{\label{tab:if_statements}
      Examples of if statements.
    }
  \end{table}

  \subsubsection{switch statements}
  The syntax for a \verb"switch" statement is
  %
  \begin{Verbatim}
    name: "switch(cond1,expr1, cond2,expr2, ...,
                                       condN,exprN, defaultExpr)"
  \end{Verbatim}
  %
  Any number of \verb"condition,expression" pairs is allowed.  Each condition
  is checked sequentially; when a condition evaluates to \verb"true", the
  variable is assigned the value of the corresponding expression, and
  subsequent conditions are not checked.  If none of the conditions is true,
  then \verb"defaultExpr" is used, if present.

  The expressions can include any valid JME syntax, including further \verb"if"
  or \verb"switch" statements.  See table~\ref{tab:switch_statements} for some
  examples.
  %
  \begin{table}[ht]
    \centering
    \begin{tabular}{ll}
      \hline
      Variable      & Value \\
      \hline
      \codevar{a}   & \verb"random(-5..5)" \\
      \codevar{ans} & \verb'"switch(a<0,10, a=0,3, a>0,-7, -99)"' \\
      \hline\hline
    \end{tabular}
    \caption{\label{tab:switch_statements}
      Examples of switch statements.
    }
  \end{table}

  \subsection{\label{sec:lists}Lists}
  A list is a finite collection of elements of any type.  A list can be defined
  explicitly using square bracket notation, \eg
  %
  \begin{Verbatim}
    a: "[1,2,3,4,5]"
  \end{Verbatim}
  %
  or implicitly using the \verb"repeat" function, which evaluates a given
  expression a given number of times, \eg
  %
  \begin{Verbatim}
    a: "repeat( random(1..10), 5 )"
  \end{Verbatim}
  %
  This example will produce a list of five numbers in the range $[1,10]$.

  A particular element of the list can be retrieved by using another form of
  the square bracket notation, \eg
  %
  \begin{Verbatim}
    a: "[1,3,5,7,9]"
    b: a[2]
  \end{Verbatim}
  %
  where indexing starts at zero.

  The \verb"map" function applies a given expression to all elements of the
  list, \eg
  %
  \begin{Verbatim}
    a: "[1,2,3,4,5]"
    b: "map(x^2, x, a)"
    c: "map(x^3, x, 1..3)"
  \end{Verbatim}
  %
  Here, the variable \codevar{b} will be the list \verb"[1,4,9,16,25]".  The
  definition of the variable \codevar{c} shows that \verb"map" also works on
  ranges, so that \codevar{c} will be the list \verb"[1,8,27]".

  Finally, the \verb"abs" function can be used to obtain the number of elements
  in the list, \eg
  %
  \begin{Verbatim}
    a: "[1,6,23]"
    b: abs(a)
  \end{Verbatim}
  %
  Here, \codevar{b} will be set to 3.

  \section{\label{sec:using_variables}Using variables}
  Once variables have been declared, they can be used in most places within a
  question definition, in particular within part answers and
  \codecontent{content} blocks (see chapter~\ref{cha:content_blocks}).  To do
  this, enclose the variable name in braces.  When the exam is displayed, the
  variable is evaluated according to its definition, and substituted in.
  Examples of variable usage are shown in table~\ref{tab:using_variables}.
  %
  \begin{table}[ht]
    \centering
    \begin{tabular}{ll}
      \hline
      Usage              & Result \\
      \hline
      \verb"{a}"         & 2 \\
      \verb"{b}"         & 3 \\
      \verb"{a}+{b}"     & 2+3 \\
      \verb"{a+b}"       & 5 \\
      \verb"{3*a+2*b}"   & 12 \\
      \verb"2*{a}+{3*b}" & 2*2+9 \\
      \hline\hline
    \end{tabular}
    \caption{\label{tab:using_variables}
      Examples of variable usage, having defined the variables
      \texttt{a: 2} and \texttt{b: 3}.
    }
  \end{table}

  \noindent\textbf{Note:} There is one exception to using braces in this way, when
  formatting \emph{displayed} mathematics using \LaTeX.  In this case, a
  slightly different syntax must be used --- see \S\ref{sec:latex_formatting}.

  \section{\label{sec:functions}Functions}
  As well as defining variables, it is possible to define your own functions,
  which can then be used in exactly the same way as the built-in functions, \eg
  \verb"cos(x)", \verb"exp(x)", \etc.  This enables you to write more
  complicated questions, than by defining variables alone.

  \subsection{Declaring functions}
  Just like variables, functions are declared in a \codeobject{functions}
  object, as a property of a \codeobject{question} object.  Consider the
  following code which defines the factorial operator.
  %
  \begin{Verbatim}
    functions: {
      factorial: {
        parameters: [ [x,number] ]
        type: number
        definition: "if(x=0,1,x*factorial(x-1))"
      }
    }
  \end{Verbatim}
  %
  Each function has a name, a list of the parameters the function takes, the
  return type of the function, and its definition.

  \textbf{Note:} There is already a built-in function to calculate the
  factorial, called \verb"fact()", so you do not need to write your own.  This
  definition is used for illustrative purposes only.

  \subsubsection{Parameters}
  The \codeprop{parameters} property defines the arguments of the function.
  Its value is an array of two-element arrays and it takes the the form
  %
  \begin{Verbatim}
    parameters: [ [arg1,type1], [arg2,type2], ..., [argN,typeN] ]
  \end{Verbatim}
  %
  Where each \verb"argi" is a dummy argument to be used in the function
  definition, and each \verb"typei" is the type of the corresponding argument
  (valid types are described in \S\ref{sec:jme_data_types}).

  \subsubsection{Definition}
  The \codeprop{definition} property defines what the function does.  Any valid
  JME syntax is allowed here.  Variables, and other functions which have
  already been defined, can be used within a function definition (notice how
  the example function even calls itself).

  The resulting type of the calculations performed in the function definition
  must match the value of the \codeprop{type} property.

  \subsection{Using functions}
  User-defined functions are called in exactly the same way as the built-in
  functions, \eg \verb"name(arg1,arg2,...argN)".

  \section{\label{sec:advice_content}Advice}
  When declared as a property of the \codeobject{question} object, the
  \codeprop{advice} property sets the content to be shown when the student
  presses the \codebutton{Reveal} button.  It is a \codecontent{content} block
  (see chapter~\ref{cha:content_blocks}), and is usually used to provide a
  fully-worked solution to the question.

\end{chapter}
